\documentclass{article}
\usepackage{textcomp}
\usepackage[utf8x]{inputenc}
\usepackage{graphicx}

\title{Laboratory No.2 at MS}
\author{gr. FAF-161 st. Polina Gore}

\begin{document}

  \begin{titlepage}

    \newcommand{\HRule}{\rule{\linewidth}{0.5mm}} % Defines a new command for the horizontal lines, change thickness here

    \center % Center everything on the page

    \textsc{\LARGE UTM}\\[1.5cm] % Name of your university/college
    \textsc{\Large MS}\\[0.5cm] % Major heading such as course name

    \HRule \\[0.4cm]
    { \huge \bfseries Laboratory No.2}\\[0.4cm] % Title of your document
    \HRule \\[1.5cm]

    \begin{minipage}{0.4\textwidth}
      \begin{flushleft} \large
        \emph{Student:}\\
        Polina \textsc{Gore} % Your name
      \end{flushleft}
    \end{minipage}
    ~
    \begin{minipage}{0.4\textwidth}
      \begin{flushright} \large
        \emph{Supervisor:} \\
        Victor \textsc{Turcanu} % Supervisor's Name
      \end{flushright}
    \end{minipage}\\[4cm]

    \vfill

    {\large \today}\\[3cm]

  \end{titlepage}

  \pagenumbering{gobble}

  \newpage
  \pagenumbering{arabic}

  \section{Densities}

    \subsection{Problem No. 1}

    Take a stick of unit length and break it into two pieces, choosing the
    breakpoint at random.\\
    Now break the longer of the two pieces at a random point.
    What is the probability that the three pieces can be used to form a triangle?\\
    Write a computer program that would simulate the experiment. Explain your results.


    \subsection{Solution}
    In this problem all we got to do is check if the broken parts can form a triangle.\\
    For this, the Function isItriangle checks if a, b, c (the catheti) form a triangle.
    It finds the longest stick and checks if the sum of the other two are less
    than the length of the longest one.

    All of that is then computed 10000 times.

  \newpage

    \subsection{Problem No.3}

    Choose independently two numbers B and C at random from the interval [-1, 1]
    with uniform distribution, and consider the quadratic equation \(X^2 + B*X + C = 0\).
    Find the probability that the roots of this equation:\\
      - are both real
      - are both positive

    \subsection{Solution}

    First of all, in a for loop of 10000 cases, we randomize a B and a C on the [-1, 1]
    interval, then we check if the obtained equations has real roots, and for the
    other case, it the roots are positive.\\
    If they are, we add 1 to the variables responsible for keeping the real roots case
    and the positive roots case.

    Thus, we obtain the probability for both cases, by dividing on the number
    of total cases.

  \newpage

    \subsection{Problem 6}

    At a mathematical conference, 10 participants are randomly seated around a circular table for meals.\\
    Using simulation, estimate the probability
    that no two people sit next to each other at both lunch and dinner.

    Can you make an intelligent conjecture for the case of n participants when n is large?

    \subsection{Solution}

    It is easier to look at the code, since it contains all the necessary information
    and comments.\\
    Anyway, as usual, for 1000 cases, we check if no 2 people are sitting next
    to each other. For this I shuffled the initial arrangement of people, and
    saved the neighbours for each person, at each meeting.\\
    Then I checked if the new neighbours dont coincide.\\
    So, for a large n, the probability of 2 people sitting to each other on 2
    different meetings is aproaching 0.


  \newpage

  \section{Random Variables}

    \subsection{Problem No.8}

    An insurance company has 1000 policies on men of age 50.\\
    The company estimates that
    the probability that a man of age 50 dies within a year is 0.01.\\
    Estimate the number of claims that
    the company can expect from beneficiaries of these men within a year.


    \subsection{Solution}

    I am not sure about this problem.\\
    Since it is only one line of code and one comment that explains the method,
    I leave it to you.

  \newpage

  \section{Important Distribution}

    \subsection{Problem No.10}

    Assume that the probability that there is a significant accident in a
    nuclear power plant during one year’s time is 0.001.\\
    If a country has 100 nuclear plants, estimate the probability
    that there is at least one Chernobyl (or Black Mesa) during a given year.


    \subsection{Solution}

    And again, I do not know how to compute this problem, so I did it manually.\\
    The explanation is inside the code.


  \newpage

  \subsection{Problem No.11}

    Let U be a uniformly distributed random variable on [0, 1].\\
    What is the probability that the equation
    \(X^2 + 4U*X + 1 = 0\) has two distinct real roots x1 and x2 ?

    \subsection{Solution}

    This problem is similar to the Problem No.3.\\
    Again, we check the roots of an equation with a random U that is on the
    interval [0, 1] if they are real and distinct and compute the probability.

  \newpage

  \section{Continuous Probability}

    \subsection{Problem No.12}

    Suppose you toss a dart at a circular target of radius 10 inches. Given that
    the dart lands in the upper half of the target, find the probability that:\\
    - it lands in the right half of the target\\
    - its distance from the center is less than 5 inches\\
    - its distance from the center is greater than 5 inches\\
    - it lands within 5 inches of the point (0, 5)\\

    \subsection{Solution}

    Here, we only use geometry to check where the darts land.\\
    For example, if it lands in the right half of the upper half of the target
    then\\ x \in [0, 10], y \in [0, \(\sqrt{10^2 - x^2}\)]\\
    For the inner circle with radius 5, it is the same, \\but x \in [-5, 5],
    y \in [0, \(\sqrt{5^2 - x^2}\)]\\
    And so on, we compute it for 1000 cases, and find the probability for each.

\end{document}
