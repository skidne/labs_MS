\documentclass{article}
\usepackage{textcomp}
\usepackage[utf8x]{inputenc}
\usepackage{graphicx}
\usepackage{xcolor}
\usepackage[most]{tcolorbox}
\usepackage{todonotes}
\usepackage{float}
\usepackage[center]{titlesec}

\begin{document}

  \begin{titlepage}

    \newcommand{\HRule}{\rule{\linewidth}{0.5mm}} % Defines a new command for the horizontal lines, change thickness here

    \center % Center everything on the page

%----------------------------------------------------------------------------------------
%	HEADING SECTIONS
%----------------------------------------------------------------------------------------

    \textsc{\LARGE Technical University of Moldova}\\[1.5cm]
    \textsc{\Large Special Mathematics}\\[0.5cm] % Major heading such as course name

%----------------------------------------------------------------------------------------
%	TITLE SECTION
%----------------------------------------------------------------------------------------

    \HRule \\[0.4cm]
    { \huge \bfseries Laboratory No.5}\\[0.4cm] % Title of your document
    \HRule \\[1.5cm]

%----------------------------------------------------------------------------------------
%	AUTHOR SECTION
%----------------------------------------------------------------------------------------

    \begin{minipage}{0.4\textwidth}
      \begin{flushleft} \large
        \emph{Author:}\\
          st. Polina \textsc{Gore}\\ gr. FAF-161 % Your name
      \end{flushleft}
    \end{minipage}
~
    \begin{minipage}{0.4\textwidth}
      \begin{flushright} \large
        \emph{Supervisor:} \\
        Victor \textsc{Țurcanu} % Supervisor's Name
      \end{flushright}
    \end{minipage}\\[2cm]

%----------------------------------------------------------------------------------------
%	DATE SECTION
%----------------------------------------------------------------------------------------

    {\large \today}\\[2cm] % Date, change the \today to a set date if you want to be precise

%----------------------------------------------------------------------------------------
%	LOGO SECTION
%----------------------------------------------------------------------------------------
    \includegraphics[width=7cm]{utm2.png}\\[1cm] % Include a department/university logo - this will require the graphicx package

%----------------------------------------------------------------------------------------

    \vfill % Fill the rest of the page with whitespace

  \end{titlepage}

%----------------------------------------------------------------------------------------
%----------------------------------------------------------------------------------------

  \newpage
  \pagenumbering{arabic}

  % -------------- PROBLEM 1------------------------------------------------------
  \restylefloat{table}
  \begin{table}[H]
    \centering
    \textbf{The adjacent matrix for the friendship graph:}\\
    \begin{tabular}{|c|c|c|c|c|c|c|c|c|c|c|c|c|c|c|c|c|c|c|c|}
      \hline
      0 & 0 & 0 & 0 & 1 & 1 & 0 & 1 & 0 & 0 & 0 & 0 & 0 & 0 & 1 & 0 & 1 & 0 & 1 & 1 \\ \hline
      0 & 0 & 1 & 0 & 0 & 0 & 0 & 0 & 0 & 0 & 0 & 0 & 1 & 0 & 0 & 0 & 0 & 0 & 0 & 0 \\ \hline
      0 & 1 & 0 & 1 & 0 & 1 & 1 & 0 & 1 & 1 & 0 & 1 & 0 & 0 & 1 & 1 & 0 & 1 & 0 & 1 \\ \hline
      0 & 0 & 1 & 0 & 0 & 1 & 1 & 0 & 0 & 0 & 1 & 0 & 0 & 0 & 0 & 1 & 0 & 1 & 0 & 1 \\ \hline
      1 & 0 & 0 & 0 & 0 & 0 & 0 & 1 & 0 & 0 & 0 & 0 & 0 & 0 & 0 & 0 & 0 & 0 & 0 & 0 \\ \hline
      1 & 0 & 1 & 1 & 0 & 0 & 1 & 0 & 0 & 1 & 1 & 1 & 1 & 1 & 0 & 0 & 0 & 1 & 0 & 1 \\ \hline
      0 & 0 & 1 & 1 & 0 & 1 & 0 & 0 & 0 & 0 & 0 & 1 & 1 & 1 & 1 & 1 & 0 & 1 & 0 & 1 \\ \hline
      1 & 0 & 0 & 0 & 1 & 0 & 0 & 0 & 0 & 0 & 0 & 0 & 0 & 0 & 0 & 0 & 0 & 0 & 0 & 0 \\ \hline
      0 & 0 & 1 & 0 & 0 & 0 & 0 & 0 & 0 & 0 & 0 & 1 & 1 & 0 & 0 & 1 & 0 & 1 & 0 & 1 \\ \hline
      0 & 0 & 1 & 0 & 0 & 1 & 0 & 0 & 0 & 0 & 1 & 0 & 1 & 1 & 1 & 0 & 0 & 0 & 0 & 0 \\ \hline
      0 & 0 & 0 & 1 & 0 & 1 & 0 & 0 & 0 & 1 & 0 & 0 & 0 & 1 & 1 & 0 & 0 & 0 & 0 & 0 \\ \hline
      0 & 0 & 1 & 0 & 0 & 1 & 1 & 0 & 1 & 0 & 0 & 0 & 0 & 0 & 0 & 1 & 0 & 0 & 0 & 0 \\ \hline
      0 & 1 & 0 & 0 & 0 & 1 & 1 & 0 & 1 & 1 & 0 & 0 & 0 & 0 & 1 & 0 & 0 & 0 & 0 & 0 \\ \hline
      0 & 0 & 0 & 0 & 0 & 1 & 1 & 0 & 0 & 1 & 1 & 0 & 0 & 0 & 0 & 0 & 0 & 0 & 0 & 0 \\ \hline
      1 & 0 & 1 & 0 & 0 & 0 & 1 & 1 & 0 & 0 & 1 & 1 & 0 & 0 & 0 & 0 & 0 & 0 & 0 & 0 \\ \hline
      0 & 0 & 1 & 1 & 0 & 0 & 1 & 0 & 1 & 0 & 0 & 1 & 0 & 0 & 0 & 0 & 1 & 0 & 0 & 0 \\ \hline
      1 & 0 & 0 & 0 & 0 & 0 & 0 & 0 & 0 & 0 & 0 & 0 & 0 & 0 & 0 & 1 & 0 & 0 & 0 & 0 \\ \hline
      0 & 0 & 1 & 1 & 0 & 1 & 1 & 0 & 1 & 0 & 0 & 0 & 0 & 0 & 0 & 0 & 0 & 0 & 0 & 0 \\ \hline
      1 & 0 & 0 & 0 & 0 & 0 & 0 & 0 & 0 & 0 & 0 & 0 & 0 & 0 & 0 & 0 & 0 & 0 & 0 & 0 \\ \hline
      1 & 0 & 1 & 1 & 0 & 1 & 1 & 0 & 1 & 0 & 0 & 0 & 0 & 0 & 0 & 0 & 0 & 0 & 0 & 0 \\ \hline
    \end{tabular}
  \end{table}

  \section{Friends}
    \centerline{Find the person with the most friends.}

  \vspace{2em}

  \subsection{Solution}
    To find the number of friends each person from this graph has, we have to sum
    the connections this person has with other people.\\ Having the adjacent matrix, it is easily done.\\
    Knowing that 1 in that matrix stands for the connection between 2 nodes and
    0 for no connection between 2 nodes, we only take a line (or a column,
    it doesn't matter, since the adjacent matrix is symmetric) that is responsible
    for a node, look at the values it has and sum them, thus, we can obtain the number
    of edges of that node, in our case it's the number of friends that a person has.\\
    By running the program (\emph{ex1\_friendly.py}), we notice that we have 2 people with
    the largest number of friends, these are:\\
    • Corrin Tally\\
    • Ellie Francese\\
    They both have 11 friends.


  \newpage

  %----------------------PROBLEM 2----------------------------------------------

  \section{Sort}
    \centerline{Sort all the people by the number of friends.}

  \vspace{3em}

  \subsection{Solution}

    In this problem we use the same approach as in the previous exercise.\\
    Just summing the 1's, and sorting them descending.

    \vspace{2em}

    \restylefloat{table}
    \begin{table}[H]
      \centering
      \textbf{The output:}\\
      \vspace{1em}
      \begin{tabular}{|c|c|}
        \hline
        \textbf{Name} & \textbf{Friends} \\ \hline
        Ellie Francese & 11 \\ \hline
        Corrin Tally   & 11 \\ \hline
        Augustine Golub & 10 \\ \hline
        Leandro Eagan  &  7 \\ \hline
        Caleb Hobby    &  7 \\ \hline
        Clarence Stalker &  6 \\ \hline
        Lili Houghton  &  6 \\ \hline
        Cruz Perna     &  6 \\ \hline
        Sammie Womac   &  6 \\ \hline
        Lorean Simcox  &  6 \\ \hline
        Pearlie Moffet &  6 \\ \hline
        Angila Ellinger &  5 \\ \hline
        Marita Tegeler &  5 \\ \hline
        Monet Mccoy    &  5 \\ \hline
        Tiny Parkhurst &  4 \\ \hline
        Alta Kennan    &  2 \\ \hline
        Otilia Laxson  &  2 \\ \hline
        Rebbecca Charlton & 2 \\ \hline
        Elinore Orsborn &  2 \\ \hline
        Jarred Marrow  &  1 \\ \hline
      \end{tabular}
    \end{table}


  \newpage

    %----------------------PROBLEM 3--------------------------------------------

    \section{Let's do ratings}
      For each person in the network, compute the ratings using Dijkstra's
      algorithm to find the shortest path from a node to another.\\

    \vspace{3em}

    \subsection{Solution}
      Solving this exercise wasn't particularly difficult, as Dijkstra's
      algorithm can be found on the Internet. \\
      So, in order to solve this problem, we have to find the shortest distances
      for every node to every other nodes, substract 1 from each distance (the points)
      and sum them together for every node.\\
      The resulting rating is as follows:\\

      \vspace{2em}

      \restylefloat{table}
      \begin{table}[H]
        \centering
        \begin{tabular}{|c|c|}
          \hline
          \textbf{Name} & \textbf{Rating} \\ \hline
          Ellie Francese    & 8 \\ \hline
          Corrin Tally      & 11 \\ \hline
          Augustine Golub   & 12 \\ \hline
          Caleb Hobby       & 13 \\ \hline
          Lorean Simcox     & 13 \\ \hline
          Pearlie Moffet    & 13 \\ \hline
          Leandro Eagan     & 15 \\ \hline
          Cruz Perna        & 16 \\ \hline
          Lili Houghton     & 17 \\ \hline
          Sammie Womac      & 17 \\ \hline
          Angila Ellinger   & 17 \\ \hline
          Marita Tegeler    & 18 \\ \hline
          Clarence Stalker  & 18 \\ \hline
          Monet Mccoy       & 20 \\ \hline
          Tiny Parkhurst    & 21 \\ \hline
          Rebbecca Charlton & 23 \\ \hline
          Alta Kennan       & 27 \\ \hline
          Elinore Orsborn   & 30 \\ \hline
          Otilia Laxson     & 30 \\ \hline
          Jarred Marrow     & 31 \\ \hline
        \end{tabular}
      \end{table}
      \newpage

  %----------------------PROBLEM 4--------------------------------------------

    \section{Influential people}
    Use the data from the previous exercise and find the new ”Rating” for each
    person by multiplying it with 0.5 of the posting rate.\\
    Please sort the people by the newly computed rating.\\

    \vspace{3em}

    \subsection{Solution}
      For this problem we have to use the data obtained in the problem No.2,
      where we found out the number of friends each person has.\\
      Then, we will compute the new rating, using the new data found in the
      file \emph{influence.txt}.\\
      Finally, we sort the people by their new rating, obtaining this:\\

      \vspace{2em}

      \restylefloat{table}
      \begin{table}[H]
        \centering
        \begin{tabular}{|c|c|}
          \hline
          \textbf{Name} & \textbf{Rating} \\ \hline
          Corrin Tally      & 47.163 \\ \hline
          Ellie Francese    & 44.825 \\ \hline
          Augustine Golub   & 28.250 \\ \hline
          Sammie Womac      & 25.500 \\ \hline
          Leandro Eagan     & 23.275 \\ \hline
          Lorean Simcox     & 20.700 \\ \hline
          Angila Ellinger   & 20.625 \\ \hline
          Marita Tegeler    & 18.812 \\ \hline
          Tiny Parkhurst    & 16.900 \\ \hline
          Lili Houghton     & 16.500 \\ \hline
          Cruz Perna        & 15.750 \\ \hline
          Monet Mccoy       & 13.875 \\ \hline
          Clarence Stalker  & 10.425 \\ \hline
          Alta Kennan       &  9.900 \\ \hline
          Pearlie Moffet    &  9.825 \\ \hline
          Caleb Hobby       &  9.713 \\ \hline
          Rebbecca Charlton &  7.700 \\ \hline
          Otilia Laxson     &  4.525 \\ \hline
          Elinore Orsborn   &  4.400 \\ \hline
          Jarred Marrow     &  3.825 \\ \hline

        \end{tabular}
      \end{table}

    \newpage
    %----------------------PROBLEM 5--------------------------------------------

    \section{Analyze your content}

    You are publishing a book and would like to promote it through the use
    of social media.\\
    The book’s title is:

    \vspace{2em}
    \textcolor{purple}{\centerline{\textbf{”From T-Rex to Justin Bieber:
    How Internet has changed the Politics, Art and cute Cats”}}}

    \vspace{2em}
    You have done some research in the world’s most popular social network
    and have found that the range of interests is stored in \emph{interests.txt} .
    Analyze your title and see what specter of interests is your book marketable to.

    \vspace{3em}

    \subsection{Solution}
      For this problem, we have to extract the \textit{interests} from the file
      mentioned above, then, after splitting the title into words, finding
      the common interests.\\
      That's all we need to find the specter of interests of this book.\\
      The results are as follows:

      \vspace{2em}

      \restylefloat{table}
      \begin{table}[H]
        \centering
        \begin{tabular}{l}
            • Internet \\ \hline
            • Art \\ \hline
            • T-Rex \\ \hline
            • Politics \\ \hline
            • Cats \\ \hline
            • Bieber~~~~~~~~~~~~~\textcolor{red}{DISCLAIMER: "Bieber" and "Music" are not related.} \\
        \end{tabular}
      \end{table}

    \newpage

    %----------------------PROBLEM 6--------------------------------------------

  \section{Promote it}
  We have provided you with a list of interests of each of these people.\\
  You can find it in \emph{interests.txt} .\\
  Considering the set of interests you have chosen, who of them would you market
  the book to?\\
  Provide us with a list of 5 people we should contact to make your book a bestseller!\\
  Please use the names found in \emph{people\_interests.txt} .\\

  \vspace{3em}

  \subsection{Solution}
    In order to obtain our top 5 promoters, we need the rating found in the
    problem No.4 and the interests (of our book's title) from the previous
    exercise.\\
    Then, using this data we compute the new rating.\\
    Our top 5 promoters are:\\

    \vspace{2em}

    \restylefloat{table}
    \begin{table}[H]
      \centering
      \begin{tabular}{|c|c|}
        \hline
        \textbf{Name} & \textbf{Rating} \\ \hline
        Ellie Francese  & 8.965 \\ \hline
        Marita Tegeler  & 7.525 \\ \hline
        Augustine Golub & 5.650 \\ \hline
        Sammie Womac    & 5.100 \\ \hline
        Leandro Eagan   & 4.655 \\ \hline
      \end{tabular}
    \end{table}

\end{document}
